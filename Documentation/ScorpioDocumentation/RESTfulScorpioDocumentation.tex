\documentclass[
10pt, % Main document font size
letterpaper, % Paper type, use 'letterpaper' for US Letter paper
oneside, % One page layout (no page indentation)
%twoside, % Two page layout (page indentation for binding and different headers)
headinclude,footinclude, % Extra spacing for the header and footer
BCOR5mm, % Binding correction
]{scrartcl}


\usepackage{mike}

%----------------------------------------------------------------------------------------
%	TITLE AND AUTHOR(S)
%----------------------------------------------------------------------------------------

\title{\normalfont\spacedallcaps{Scorpio REST API}} % The article title

\author{\spacedlowsmallcaps{Michael Meding* , mmeding@outsmartinc.com}} % The article author(s) - author affiliations need to be specified in the AUTHOR AFFILIATIONS block

\date{} % An optional date to appear under the author(s)

%----------------------------------------------------------------------------------------

\begin{document}

\maketitle % Print the title and abstract box

\setcounter{tocdepth}{2} % Set the depth of the table of contents to show sections and subsections only

\tableofcontents % Print the contents section

\thispagestyle{empty} % Removes page numbering from the first page

%----------------------------------------------------------------------------------------
%	ARTICLE CONTENTS
%----------------------------------------------------------------------------------------
 
\section*{Abstract}
This article details the usage of all the Outsmart web interface RESTful API. This API provides the functionality needed to interface with all data necessary for the front end website.
 This article will include details regarding the usage and implementation of all available methods supported by the Outsmart app server. For the purposes of this article ``BASE-URL'' should be 
 substituted with ``http://dev.outsmartinc.com/scorpio`` as it will be used often as a prefix for the following URL's
 

%------------------------------------------------

\section{AuthService}

%------------------------------------------------

\subsection{\textbf{ping}}
%Code Snippet
\begin{lstlisting}
#Curl script for ping
echo Ping

curl -v -H "Accept: application/json" \
BASE-URL/auth/ping

\end{lstlisting}

%Output
\paragraph{Output (JSON)}~
\begin{lstlisting}[language=json]
{"status":"OK","version":1.0,"errorCode":0,"errorMsg":null,"fieldErrors":null,"data":null} 
\end{lstlisting}


%Url Pattern
\paragraph{URL Pattern} 
~\newline
BASE-URL/auth/ping

%Explanation
\paragraph{Explanation} Ping is a straightforward simple request-response method. It returns a 200 OK response when called.

%------------------------------------------------

\subsection{\textbf{loginAsMember}}

\begin{lstlisting}
#Curl script for login
customer="OutSmart Power Systems"
username="uwe"
password="ccaes1"

echo Login with DTO

printf '{"customer":"%s","username":"%s",
"password":"%s"}' "$customer" $username $password > scorpio_login.data 

curl -v -H "Accept: application/json" -H "Content-Type: application/json" 
  --cookie-jar cookie_jar.txt --cookie cookie_jar.txt \
  -X POST -d @scorpio_login.data \
  http://dev.outsmartinc.com/scorpio/ \
  auth/loginAsMember

\end{lstlisting}

%Output
\paragraph{Output (Header Text)}~
\begin{lstlisting}
- Replaced cookie \newline OSSESSIONID="94a3b7feab524e9204dcb84a8076a7db" for domain dev.outsmartinc.com, path /, expire 1404331216 \linebreak
- Set-Cookie: \newline OSSESSIONID=94a3b7feab524e9204dcb84a8076a7db; Version=1; Path=/; Max-Age=100
\end{lstlisting}

%Url Pattern
\paragraph{URL Pattern} 
~\newline
BASE-URL/auth/loginAsMember

%Explanation
\paragraph{Explanation} This method is the heart of AuthServie and allows the given user to login. Customer, username and password fields must match with a corresponding record in the database. Upon sucessful login a cookie named OSESSIONID is given back as a part of the response header and must be included with all other service calls. This cookie with its corresponding hash changes after each service call and is used in tracking the session and permissions of the user. If this cookie expires, or 15 minutes of server time elapses, then the session becomes invalidated and the user will no longer be able to make service calls without logging in again.


%------------------------------------------------

\section{LocationManagement}

%------------------------------------------------

\subsection{\textbf{getAllLocations}}
\begin{lstlisting}[language=Java]
// Java test code snippet
URI uri = new URI(baseUri + "LocationManagement/getAllLocations");
System.out.println("URI: " + uri.toString());

WebTarget target = client.target(uri);

String response = target.request(MediaType.APPLICATION_JSON)
  .header("Origin", "*") // must be included
  .post(null, String.class);
  
// create usable JSON from String
JSONObject jo = DefaultJSONFactory.getInstance()
  .jsonObject(response);
System.out.println("data: " + jo.toString());
\end{lstlisting}

%Output
\paragraph{Output (JSON)}~
\begin{lstlisting}[language=json]
{"status":"OK","data":[{"postalCountry":"USA","tzOffset":-14400000,"weatherStationRef":1332878257808,"postalState":"MA","postalZip":"02054","id":750,"customerId":13,"jsonobjectName":"LocationDetails","name":"Millis","acquisitionZone":"DEFAULT","postalCity":"Millis","active":true,"longitude":-71.354,"postalStreet1":"6 Milliston Rd","latitude":42.167,"timeZoneName":"America/New_York","postalStreet3":null,"postalStreet2":null},{"postalCountry":"USA","tzOffset":-14400000,"weatherStationRef":1332878257808,"postalState":"MA","postalZip":null,"id":751,"customerId":13,"jsonobjectName":"LocationDetails","name":"Easton","acquisitionZone":"DEFAULT","postalCity":"Easton","active":true,"longitude":-71.095,"postalStreet1":null,"latitude":42.088,"timeZoneName":"America/New_York","postalStreet3":null,"postalStreet2":null}],"version":1}
\end{lstlisting}


%Url Pattern
\paragraph{URL Pattern} 
~\newline
BASE-URL/LocationManagement/getAllLocations

%Explanation
\paragraph{Explanation} The JSONString in the above output contains all the relevant information stored in an array called ''data''. Additionaly the code snippet above was written as a part of a JUnit test that included a sample login that would yeald this data. All methods from here will assume a valid login and session cookie have been issued before the method was called. This particular method returns a list of all valid locations for a given customer (the one you are logged in as) and includes all relevant information for that location. 


%------------------------------------------------

\subsection{\textbf{getLoadDetails}}

\begin{lstlisting}[language=Java]
// Java test code snippet
URI uri = new URI(baseUri + "LocationManagement/getLoadDetails");
long now = System.currentTimeMillis();

//ARGS
LocationDTO dto = new LocationDTO();
dto.setLocationId(750);
dto.setMeasurementPointId(1398368055232L);
dto.setFrom((now - (10 * 60 * 60 * 1000)));
dto.setTo(now);
dto.setDataView("min");

WebTarget target = client.target(uri);

String response = target.request(MediaType.APPLICATION_JSON)
  .header("Origin", "*") //important
  .post(Entity.json(dto), String.class);
JSONObject jo = DefaultJSONFactory.getInstance()
.jsonObject(response);
\end{lstlisting}

%Output
\paragraph{Output (JSON)} ~
\begin{lstlisting}[language=json]
{"nphases":3,"location":"Uncategorized","thisMonthUsage":0,"periodPeak":"746","hasBreaker":true,"peakCapacity":"0","endUse":"HVAC/Kitchen Hood Fan","periodTotal":"7437","electrical":"Panel KPL1 & 2 (208)/1 KHEF1","energyMates":[{"mac":"0x1240003BF/C"},{"mac":"0x1240003BF/B"},{"mac":"0x1240003BF/A"}],"ytdUsage":0,"name":"KHEF1","loadBalance":[1.682885348905083E11,1.7608065969601843E11,2.1944963546792365E11],"breaker":"1","breakerCapacity":"15","peakCapacityTime":-1234,"breakerPanel":"Panel KPL1 & 2 (208)"}
\end{lstlisting}

%Url Pattern
\paragraph{URL Pattern} 
~\newline
BASE-URL/LocationManagement/getLoadDetails

%Explanation
\paragraph{Explanation} In the code snippet above I included a DTO which is a ``Data Transfer Object''. This object when passed as an Entity in the request gets serialized and transfered as a JSONObject to the REST call. When recieved in the REST service the object is reconstructed and used as a plain old java object (POJO). This method call given the correct parameters (as DTO or JSON argument) will return the current load data of then given measurementPointId.


%------------------------------------------------

\section{GenericsManagement}

%------------------------------------------------

\subsection{\textbf{getRootTags}}

\begin{lstlisting}[language=Java]
// Java test code snippet
URI uri = new URI(baseUri + "GenericsManagement/getRootTags");

//ARGS
long locationId = 202L; //existing tag
GenericsManagementDTO dto = new GenericsManagementDTO(locationId);

WebTarget target = client.target(uri);

String response = target.request(MediaType.APPLICATION_JSON)
  .header("Origin", "*") //important
  .post(Entity.json(dto), String.class);
JSONObject jo = DefaultJSONFactory.getInstance().jsonObject(response);
\end{lstlisting}

%Output
\paragraph{Output (JSON)} ~
\begin{lstlisting}[language=json]
data: {"data":"<?xml version=\"1.0\" encoding=\"UTF-8\" standalone=\"yes\"?><deviceTag id=\"1380746460877\" name=\"device_tags\" locationId=\"202\"><deviceTag id=\"1380746460880\" name=\"Electrical\" locationId=\"202\"/><deviceTag id=\"1380746461403\" name=\"End Use\" locationId=\"202\"/><deviceTag id=\"1380746461481\" name=\"Location\" locationId=\"202\"/><\/deviceTag>"}
\end{lstlisting}

%Url Pattern
\paragraph{URL Pattern} 
~\newline
BASE-URL/GenericsManagement/getRootTags

%Explanation
\paragraph{Explanation} A short JSON object that only returns the base parameters of each of the devices at a given locationId.


%------------------------------------------------

\subsection{\textbf{getSubTags}}

\begin{lstlisting}[language=Java]
// Java test code snippet
URI uri = new URI(baseUri + "GenericsManagement/getSubTags");

//ARGS
long tagId = 1381322867514L; 
boolean includeDetails = false;
GenericsManagementDTO dto = new GenericsManagementDTO(locationId); // searching for location 202
dto.setIncludeDetails(includeDetails);

WebTarget target = client.target(uri);
//Get JSON string
String response = target.request(MediaType.APPLICATION_JSON)
  .header("Origin", "*") //important
  .post(Entity.json(dto), String.class);
// Make object
JSONObject jo = DefaultJSONFactory.getInstance().jsonObject(response);
\end{lstlisting}

%Output
\paragraph{Output}~
\begin{lstlisting}[language=json]
{"data":"<?xml version=\"1.0\" encoding=\"UTF-8\" standalone=\"yes\"?><deviceTag id=\"1381322867351\" displayId=\"1381322867514\" name=\"device_tags\" locationId=\"24\"><deviceTag id=\"1381322867355\" name=\"Electrical\" locationId=\"24\"><deviceTag id=\"1381322867356\" name=\"Panel A\" locationId=\"24\"><deviceTag id=\"1381322867358\" name=\"10 Furnace\" locationId=\"24\"/><deviceTag id=\"1381322867359\" name=\"11 Lights - Server Room, Store Room\" locationId=\"24\"/>.....}
\end{lstlisting}

%Url Pattern
\paragraph{URL Pattern} 
~\newline
BASE-URL/GenericsManagement/getSubTags

%Explanation
\paragraph{Explanation} This method returns a lengthy JSONString that includes all devices and their relevant information for a given locationId.


%------------------------------------------------

\subsection{\textbf{getUncategorizedTags}}

\begin{lstlisting}[language=Java]
// Java test code snippet
URI uri = new URI(baseUri + "GenericsManagement/getUncategorizedTags");

//ARGS
long locationId = 1381322867514L;
GenericsManagementDTO dto = new GenericsManagementDTO(locationId); 

WebTarget target = client.target(uri);

String response = target.request(MediaType.APPLICATION_JSON)
  .header("Origin", "*") //important
  .post(Entity.json(dto), String.class);
JSONObject jo = DefaultJSONFactory.getInstance().jsonObject(response);
\end{lstlisting}

%Output
\paragraph{Output}~
\begin{lstlisting}[language=json]
 {"data":"<?xml version=\"1.0\" encoding=\"UTF-8\" standalone=\"yes\"?><deviceTag id=\"0\" name=\"Uncategorized\" locationId=\"24\"><deviceNode idref=\"1381322858725\" name=\"0x11E00009A/A\" isRef=\"true\"/><deviceNode idref=\"1381322866050\" name=\"0x11E0000C3/A\" isRef=\"true\"/><deviceNode idref=\"1381322865941\" name=\"0x11E0000C7/A\" isRef=\"true\"/><deviceNode idref=\"1381322865840\" name=\"0x11E0000C8/A\" isRef=\"true\"/><deviceNode idref=\"1381322858979\" name=\"0x11E000131/A\" isRef=\"true\"/><deviceNode idref=\"1381322859106\" name=\"0x11E00019C/A\" isRef=\"true\"/><deviceNode idref=\"1381322856258\" name=\"0x120000000/A\" isRef=\"true\"/><deviceNode....}
\end{lstlisting}


%Url Pattern
\paragraph{URL Pattern} 
~\newline
BASE-URL/GenericsManagement/getUncategorizedTags

%Explanation
\paragraph{Explanation} Returns any device tags not specified in either getRootTags or getSubTags.


%------------------------------------------------

\section{EquipmentManagement}

%------------------------------------------------

\subsection{\textbf{fetchAllEquipmentTags}}

\begin{lstlisting}[language=Java]
// Java test code snippet
URI uri = new URI(baseUri + "EquipmentManagement/fetchAllEquipmentTags");        

//ARGS
long locationId = 24; // existing tag
EquipmentManagementDTO dto = new EquipmentManagementDTO();
dto.setLocationId(locationId);

WebTarget target = client.target(uri);

String response = target.request(MediaType.APPLICATION_JSON)
  .header("Origin", "*") //important
  .post(Entity.json(dto), String.class);
JSONObject jo = DefaultJSONFactory.getInstance().jsonObject(response);
\end{lstlisting}

%Output
\paragraph{Output (JSON)} ~
\begin{lstlisting}[language=json]
{"data":["Furnace","Kitchen App","Lights","Office","Outlets","RTU"]}
\end{lstlisting}

%Url Pattern
\paragraph{URL Pattern} 
~\newline
BASE-URL/EquipmentManagement/fetchAllEquipmentTags

%Explanation
\paragraph{Explanation} This method returns the different tags that are attached to the equipment at a given locationId


%------------------------------------------------

\section{EnergyManagement}

%------------------------------------------------

\subsection{\textbf{fetchBaselineData}}

\begin{lstlisting}[language=Java]
// Java test code snippet
 URI uri = new URI(baseUri + "EnergyManagement/fetchBaselineData");

long now = System.currentTimeMillis();

//ARGS
EnergyManagementDTO dto = new EnergyManagementDTO();
dto.setLocationId(750);
dto.setTagId(1398368055532L);
dto.setFrom(new Timestamp(now - (10 * 60 * 60 * 1000)));
dto.setTo(new Timestamp(now));
dto.setDataView("min");

WebTarget target = client.target(uri);

String response = target.request(MediaType.APPLICATION_JSON)
  .header("Origin", "*") //important
  .post(Entity.json(dto), String.class);
JSONObject jo = DefaultJSONFactory.getInstance().jsonObject(response);
JSONArray array = jo.getJSONArray("data");        
\end{lstlisting}

%Output
\paragraph{Output (JSON)} ~
\begin{lstlisting}[language=json]
{"category_minPeriod":"ss","category_parseDates":true,"category_fieldName":"x","data":[{"y01":377292.4484375,"y02":438328.259375,"y00":309420.7078125,"x":1404372060000},{"y01":377292.4484375,"y02":438328.259375,"y00":309420.7078125,"x":1404372120000},{"y01":377292.4484375,"y02":438328.259375,"y00":309420.7078125,"x":1404372180000},{"y01":377292.4484375,"y02":438328.259375,"y00":309420.7078125,"x":1404372240000},{"y01":377292.4484375,"y02":438328.259375,"y00":309420.7078125,"x":1404372300000},{"y01":377292.4484375,"y02":438328.259375,"y00":309420.7078125,"x":1404372360000},{"y01":377292.4484375,"y02":438328.259375,"y00":309420.7078125,"x":1404372420000}....}
\end{lstlisting}

%Url Pattern
\paragraph{URL Pattern} 
~\newline
BASE-URL/EnergyManagement/fetchBaselineData

%Explanation
\paragraph{Explanation} Fetches the baseline data of a given device tagId. A valid time range must also be given. This method may return a large set of values based on the time range.

%------------------------------------------------

\subsection{\textbf{fetchCurrentPowerData}}

\begin{lstlisting}[language=Java]
// Java test code snippet
URI uri = new URI(baseUri + "EnergyManagement/fetchCurrentPowerData");        

long now = System.currentTimeMillis();

//ARGS
EnergyManagementDTO dto = new EnergyManagementDTO();
dto.setLocationId(750);
dto.setTagId(1398368055532L);
dto.setFrom(new Timestamp(now - (10 * 60 * 60 * 1000)));
dto.setTo(new Timestamp(now));
dto.setDataView("min");

WebTarget target = client.target(uri);

String response = target.request(MediaType.APPLICATION_JSON)
  .header("Origin", "*")
  .post(Entity.json(dto), String.class);
JSONObject jo = DefaultJSONFactory.getInstance().jsonObject(response);
JSONArray array = jo.getJSONArray("data");        
\end{lstlisting}

%Output
\paragraph{Output (JSON)} ~
\begin{lstlisting}[language=json]
{"category_minPeriod":"ss","category_parseDates":true,"category_fieldName":"x","data":[{"y00":316142.0911458333,"x":1404372060000},{"y00":316142.0911458333,"x":1404372120000},{"y00":318033.68072916666,"x":1404372180000},{"y00":320159.5260416666,"x":1404372240000},{"y00":316298.3947916667,"x":1404372300000},{"y00":316549.3114583333,"x":1404372360000},{"y00":316329.03489583335,"x":1404372420000},{"y00":311928.7729166667,"x":1404372480000},{"y00":313797.12604166666,"x":1404372540000},{"y00":313572.5526041667,"x":1404372600000},{"y00":313185.6244791667,"x":1404372660000},{"y00":317352.1666666667,"x":1404372720000},{"y00":324432.31614583335,"x":1404372780000}....}
\end{lstlisting}

%Url Pattern
\paragraph{URL Pattern} 
~\newline
BASE-URL/EnergyManagement/fetchCurrentPowerData

%Explanation
\paragraph{Explanation} This method will return the power data for the given time range of the device given. Can also be quite lengthy based on the size of the time range.

%------------------------------------------------

\subsection{\textbf{fetchEnergyData}}

\begin{lstlisting}[language=Java]
// Java test code snippet
URI uri = new URI(baseUri + "EnergyManagement/fetchEnergyData");
System.out.println("URI: " + uri.toString());

long now = System.currentTimeMillis();

//ARGS
EnergyManagementDTO dto = new EnergyManagementDTO();
dto.setLocationId(750);
dto.setFrom(new Timestamp(now - (10 * 60 * 60 * 1000)));
dto.setTo(new Timestamp(now));
dto.setDataView("min");

WebTarget target = client.target(uri);

String response = target.request(MediaType.APPLICATION_JSON)
  .header("Origin", "*")
  .post(Entity.json(dto), String.class);
JSONObject jo = DefaultJSONFactory.getInstance().jsonObject(response);
JSONArray array = jo.getJSONArray("data");
\end{lstlisting}

%Output
\paragraph{Output (JSON)} ~
\begin{lstlisting}[language=json]
{"category_minPeriod":"ss","category_parseDates":true,"category_fieldName":"x","data":[{"y00":316142.0911458333,"x":1404372060000},{"y00":316142.0911458333,"x":1404372120000},{"y00":318033.68072916666,"x":1404372180000},{"y00":320159.5260416666,"x":1404372240000},{"y00":316298.3947916667,"x":1404372300000},{"y00":316549.3114583333,"x":1404372360000},{"y00":316329.03489583335,"x":1404372420000},{"y00":311928.7729166667,"x":1404372480000},{"y00":313797.12604166666,"x":1404372540000},{"y00":313572.5526041667,"x":1404372600000},{"y00":313185.6244791667,"x":1404372660000},{"y00":317352.1666666667,"x":1404372720000},{"y00":324432.31614583335,"x":1404372780000}...}
\end{lstlisting}

%Url Pattern
\paragraph{URL Pattern} 
~\newline
BASE-URL/EnergyManagement/fetchEnergyData

%Explanation
\paragraph{Explanation} Fetches the raw energy data for a locationId.

%------------------------------------------------

\subsection{\textbf{fetchTemperatureData}}

\begin{lstlisting}[language=Java]
// Java test code snippet
URI uri = new URI(baseUri + "EnergyManagement/fetchTemperatureData");

long now = System.currentTimeMillis();

EnergyManagementDTO dto = new EnergyManagementDTO();
dto.setLocationId(750);
dto.setFrom(new Timestamp(now - (10 * 60 * 60 * 1000)));
dto.setTo(new Timestamp(now));
dto.setDataView("min");
dto.setTempUnit("F");

WebTarget target = client.target(uri);

String response = target.request(MediaType.APPLICATION_JSON)
  .header("Origin", "*")
  .post(Entity.json(dto), String.class);
JSONObject jo = DefaultJSONFactory.getInstance().jsonObject(response);
JSONArray array = jo.getJSONArray("data");        
\end{lstlisting}

%Output
\paragraph{Output (JSON)} ~
\begin{lstlisting}[language=json]
{"category_minPeriod":"ss","category_parseDates":true,"category_fieldName":"x","data":[{"y00":74.67189107076695,"x":1404372060000},{"y00":74.67169099666543,"x":1404372120000},{"y00":74.6714909225639,"x":1404372180000},{"y00":74.67129084846239,"x":1404372240000},{"y00":74.67109077436086,"x":1404372300000},{"y00":74.67089070025935,"x":1404372360000},{"y00":74.67069062615784,"x":1404372420000},{"y00":74.67049055205632,"x":1404372480000},{"y00":74.6702904779548,"x":1404372540000},{"y00":74.67009040385327,"x":1404372600000},{"y00":74.66989032975175,"x":1404372660000},{"y00":74.66969025565024,"x":1404372720000},{"y00":74.66949018154872,"x":1404372780000}...}
\end{lstlisting}

%Url Pattern
\paragraph{URL Pattern} 
~\newline
BASE-URL/EnergyManagement/fetchTemperatureData

%Explanation
\paragraph{Explanation} As with the previous methods this returns a large JSONString that contains raw temperature data.

%------------------------------------------------

\subsection{\textbf{fetchUseData}}

\begin{lstlisting}[language=Java]
// Java test code snippet
 URI uri = new URI(baseUri + "EnergyManagement/fetchUseData");        

long now = System.currentTimeMillis();

//"from":1404100800000,"to":1404152580000,"rootTagId":"1398368055532","locationId":750
EnergyManagementDTO dto = new EnergyManagementDTO();
dto.setLocationId(750);
dto.setRootTagId(1398368055532L);
dto.setFrom(new Timestamp(now - (10 * 60 * 60 * 1000)));
dto.setTo(new Timestamp(now));
dto.setDataView("min");

WebTarget target = client.target(uri);

String response = target.request(MediaType.APPLICATION_JSON)
  .header("Origin", "*")
  .post(Entity.json(dto), String.class);
JSONObject jo = DefaultJSONFactory.getInstance().jsonObject(response);
JSONArray array = jo.getJSONArray("data");
\end{lstlisting}

%Output
\paragraph{Output (JSON)} ~
\begin{lstlisting}[language=json]
{"displayType":"rollup","data":[{"id":1398368055569,"uom":"Wh","name":"Refrigeration","value":"2247339","isTag":true},{"id":1398368055538,"uom":"Wh","name":"HVAC","value":"609211","isTag":true},{"id":1398368055550,"uom":"Wh","name":"Lighting","value":"451544","isTag":true},{"id":1398368055554,"uom":"Wh","name":"Mixed Use","value":"303588","isTag":true},{"id":1398368055533,"uom":"Wh","name":"Food Prep","value":"292844","isTag":true},{"id":1398368055557,"uom":"Wh","name":"Other","value":"71193","isTag":true}]}
\end{lstlisting}

%Url Pattern
\paragraph{URL Pattern} 
~\newline
BASE-URL/EnergyManagement/fetchUseData

%Explanation
\paragraph{Explanation} Repetative yet? This method returns the use data for the rootTagId in a given time range.

%------------------------------------------------

\subsection{\textbf{fetchTopLoads}}

\begin{lstlisting}[language=Java]
// Java test code snippet
URI uri = new URI(baseUri + "EnergyManagement/fetchTopLoads");        

long now = System.currentTimeMillis();

EnergyManagementDTO dto = new EnergyManagementDTO();
dto.setLocationId(750);
dto.setTagId(1398368055532L);
dto.setFrom(new Timestamp(now - (10 * 60 * 60 * 1000)));
dto.setTo(new Timestamp(now));
dto.setDataView("min");
dto.setnLoads(2);

WebTarget target = client.target(uri);

String response = target.request(MediaType.APPLICATION_JSON)
  .header("Origin", "*")
  .post(Entity.json(dto), String.class);
JSONObject jo = DefaultJSONFactory.getInstance().jsonObject(response);
JSONArray array = jo.getJSONArray("data");
\end{lstlisting}

%Output
\paragraph{Output (JSON)} ~
\begin{lstlisting}[language=json]
{"displayType":"rollup","data":[{"id":1398368041755,"uom":"Wh","name":"Rack B - (Main Switch Gear (Left+Right) - R1)","value":"478899","isTag":false},{"id":1398368039281,"uom":"Wh","name":"RTU-1 - (Main Switch Gear (Left+Right) - R2)","value":"472782","isTag":false},{"id":1398368003263,"uom":"Wh","name":"Panel PPA - (Main Switch Gear (Left+Right) - L3)","value":"325569","isTag":false},{"id":1398368030859,"uom":"Wh","name":"Panel NLC Lighting - (Main Switch Gear (Left+Right) - R4)","value":"285725","isTag":false},{"id":1398368007959,"uom":"Wh","name":"Comp B6 - (Rack B, Med Temp, Glycol (480) - B6)","value":"280447","isTag":false},{"id":1398368015632,"uom":"Wh","name":"Rack A - (Main Switch Gear (Left+Right) - R5)","value":"279993","isTag":false},{"id":1398368035704,"uom":"Wh","name":"Comp B4 - (Rack B, Med Temp, Glycol (480) - B4)","value":"264469","isTag":false},{"id":1398368010776,"uom":"Wh","name":"ATS - (Main Switch Gear (Left+Right) - L4)","value":"261880","isTag":false},{"id":1398368026413,"uom":"Wh","name":"Comp B5 - (Rack B, Med Temp, Glycol (480) - B5)","value":"257798","isTag":false},{"id":1398368017814,"uom":"Wh","name":"main fan (M10) - (RTU 1 (480) - F1)","value":"225856","isTag":false},{"id":1398368034815,"uom":"Wh","name":"Panel RF (Case Fans) - (208V Distribution Panel (DPA) - 7)","value":"190088","isTag":false}...}
\end{lstlisting}

%Url Pattern
\paragraph{URL Pattern} 
~\newline
BASE-URL/EnergyManagement/fetchTopLoads

%Explanation
\paragraph{Explanation} This is a particularly time consuming REST call. It averages between 15-20 seconds because it has to crunch so much data which can clearly be seen above.

\end{document}
