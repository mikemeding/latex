\documentclass[
10pt, % Main document font size
letterpaper, % Paper type, use 'letterpaper' for US Letter paper
oneside, % One page layout (no page indentation)
%twoside, % Two page layout (page indentation for binding and different headers)
headinclude,footinclude, % Extra spacing for the header and footer
BCOR5mm, % Binding correction
]{scrartcl}


\usepackage{mike}

%----------------------------------------------------------------------------------------
%	TITLE AND AUTHOR(S)
%----------------------------------------------------------------------------------------

\title{\normalfont\spacedallcaps{Installmate REST API}} % The article title

\author{\spacedlowsmallcaps{Michael Meding* , mmeding@outsmartinc.com}} % The article author(s) - author affiliations need to be specified in the AUTHOR AFFILIATIONS block

\date{} % An optional date to appear under the author(s)

%----------------------------------------------------------------------------------------

\begin{document}

\maketitle % Print the title and abstract box

\setcounter{tocdepth}{2} % Set the depth of the table of contents to show sections and subsections only

\tableofcontents % Print the contents section

\thispagestyle{empty} % Removes page numbering from the first page

%----------------------------------------------------------------------------------------
%	ARTICLE CONTENTS
%----------------------------------------------------------------------------------------
 
\section*{Abstract}

 This article will include details regarding the usage and implementation of all available methods supported by the Outsmart app server. For the purposes of this article ``BASE-URL'' should be 
 substituted with "http://ia.out smartinc.com/" as it will be used often as a prefix for the following URL's.
 In addition this documentation only includes one class with several services which have not been fully implemented at the time of this writing. They include, getOpStatus, pingMac and savePanelData.
 


%------------------------------------------------

\section{AuthService}

%------------------------------------------------

\subsection{\textbf{fetchModes}}
%Code Snippet
\begin{lstlisting}
#Curl script
curl -H "Origin:*" -v -X POST
	BASE-URL/installmate/fetchModes
\end{lstlisting}

%Output
\paragraph{Output (JSON)}~
\begin{lstlisting}[language=json]
{"unavaiableCalls":["fetchPanelReferences","fetchPanelData"],"haveELR":true}
\end{lstlisting}


%Url Pattern
\paragraph{URL Pattern} 
~\newline
BASE-URL/installmate/fetchModes

%Explanation
\paragraph{Explanation} Simple call that returns the modes "elr/ops". Including the -H "Origin:*" is nessary for cross origin resource sharing. 


%------------------------------------------------

\subsection{\textbf{fetchCustomers}}

\begin{lstlisting}
#Curl script
cu_{}rl -H "Origin:*" -v -X POST 
	BASE-URL/installmate/fetchCustomers
\end{lstlisting}

%Output
\paragraph{Output (Header Text)}~
\begin{lstlisting}[language=json]
{"status":"OK","version":1.0,"list":["OutSMart Ppwer SUstems","OutSmart Power Systems","laber"]}
\end{lstlisting}

%Url Pattern
\paragraph{URL Pattern} 
~\newline
BASE-URL/installmate/fetchCustomers

%Explanation
\paragraph{Explanation} This method should be called fetchAllCustomers. It returns all the customers it has stored as a list of strings.


%------------------------------------------------

\subsection{\textbf{fetchLocations}}

\begin{lstlisting}
#Curl script
curl -v -H "Accept: application/json" 
	-H "Content-Type: application/json" 
	-X POST -d "{\"customerName\":\"laber\"}" 
	BASE-URL/installmate/fetchLocations
\end{lstlisting}

%Output
\paragraph{Output (Header Text)}~
\begin{lstlisting}[language=json]
{"status":"OK","version":1.0,"list":["glork","lall"]}
\end{lstlisting}

%Url Pattern
\paragraph{URL Pattern} 
~\newline
BASE-URL/installmate/fetchLocations

%Explanation
\paragraph{Explanation} I figure by this point you must understand what is going on. The only differance between this method and the ones prior is that this one has a single argument that must be passed as JSON.


%------------------------------------------------

\subsection{\textbf{fetchPanelData}}

\begin{lstlisting}
#Curl script
curl -v -H "Accept: application/json" 
	-H "Content-Type: application/json" 
	-X POST -d "{\"customerName\":\"laber\"
	,\"locationName\":\"lall\"}" 
	BASE-URL/installmate/fetchPanelData
\end{lstlisting}

%Output
\paragraph{Output (Header Text)}~
\begin{lstlisting}[language=json]
{"anObject":"aString"}
\end{lstlisting}

%Url Pattern
\paragraph{URL Pattern} 
~\newline
BASE-URL/installmate/fetchPanelData

%Explanation
\paragraph{Explanation} Again, if you understand everything else up to this point then you should have no problem figuring out what is going on here.  


%------------------------------------------------

\subsection{\textbf{fetchPanelData}}

\begin{lstlisting}
#Curl script
curl -v -H "Accept: application/json" 
	-H "Content-Type: application/json" 
	-X POST -d "{\"customerName\":\"laber\"
	,\"locationName\":\"lall\"}"
	BASE-URL/installmate/fetchPanelData
\end{lstlisting}

%Output
\paragraph{Output (Header Text)}~
\begin{lstlisting}[language=json]
{"anObject":"aString"}
\end{lstlisting}

%Url Pattern
\paragraph{URL Pattern} 
~\newline
BASE-URL/installmate/fetchPanelData

%Explanation
\paragraph{Explanation} Again, if you understand everything else up to this point then you should have no problem figuring out what is going on here.  


%------------------------------------------------

\subsection{\textbf{savePanelData}}

\begin{lstlisting}
#Curl script
curl -v -H "Accept: application/json" 
	-H "Content-Type: application/json" 
	-X POST -d "{\"customerName\":\"laber\"
	,\"locationName\":\"lall\",
	,\"panelData\":\"\"{\"variable\":\"some_data\"}\"\",}" 
	BASE-URL/installmate/savePanelData
\end{lstlisting}

%Output
\paragraph{Output (Header Text)}~
\begin{lstlisting}[language=json]
{"status":"OK","version":1.0}
\end{lstlisting}

%Url Pattern
\paragraph{URL Pattern} 
~\newline
BASE-URL/installmate/savePanelData

%Explanation
\paragraph{Explanation} You put the JSON in, some magic happens and BLAM! Your stuff gets saved. Amazing!

\end{document}
