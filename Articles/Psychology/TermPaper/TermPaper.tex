\documentclass[a4paper,man,natbib]{apa6}

\makeatletter
\renewcommand{\paragraph}{}

\usepackage[english]{babel}
\usepackage[utf8x]{inputenc}
\usepackage{amsmath}
\usepackage[colorinlistoftodos]{todonotes}

\title{Video Game Addiction}
\shorttitle{Video Game Addiction}
\author{Michael Meding}
\affiliation{University of Massachusetts Lowell}

<<<<<<< HEAD
\abstract{
This paper details research pertaining to video game addiction and its effects. My findings during this research show that video game addiction is something that is not being taken seriously enough even though its effects on teens and young adults has been proven in other studies. This paper details some of the findings from those studies as well as several edge cases that have made national news.
}
=======
\abstract{blahhhh}
>>>>>>> a4ea45151d9cc614ed59d22f1c219986701ec1c5
\begin{document}
\maketitle

\section{Introduction}

\paragraph{}	Video games are a popular pastime for children and young adults, but for some it becomes more than just a hobby. In the past decade, there have been significantly more documented problems with video game addiction. However, most people do not understand why this addiction is a problem. As video games, especially online ones, gain a bigger fan base, video game addiction should be recognized in the same way as other addictions are, such as gambling or alcohol.

\section{Signs and Symptoms}
\paragraph{}Addiction, as defined by the American Psychiatric Association, is a chronic brain disease that cause compulsive behaviour despite the consequences. To classify someone as being truly addicted to video games, there are several signs that people must be aware of. Dr. Brent Conrad, a clinical psychologist, has written guides for parents explaining the symptoms of video game addiction. A few of the signs an addict will show are denial, loss of interest in other activities, and depression. Denial is a major factor in being able to determine how addicted someone has become, “young people who are addicted to computer games (especially teenagers) may lie to parents and family members about how often they play,” (Conrad). Someone who is heavily addicted to a game will often lie about their playing time because if they admit that they play too much they may lose the privilege to play the game. The level of denial can also worsen if the person has the game system in their room and can play without a family member or friend noticing. The second noticeable sign is the loss of interest in other activities. Conrad suggests that “if a child’s only interest is computer games, this is a very good indication that his or her video game habits are becoming excessive or unhealthy,” (Conrad). Sudden loss of interest in other activities can be used as an indicator to many addictions and as a video game addiction is so time consuming this becomes obvious at an early stage. The last important sign Conrad gives with identifying video game addiction is depression. The player “may spend much of their time thinking about how to regain access to the computer, claim that all other activities are ‘boring,’ and have difficulty concentrating and focusing because they are daydreaming about the game,” (Conrad). Online multiplayer video games are designed to create an emotional escape from the real world. If an addicted gamer is forcibly removed from his or her game it is likely that he or she will become anxious or depressed as the game has become their only escape from having to face real world emotional issues.

\section{Dangers}
<<<<<<< HEAD
\paragraph{}There are also several dangers of becoming addicted to video games and when left untreated can result in physical harm and, in extreme cases, death. One of the dangers of video game addiction is eating disorders. For decades, unhealthy eating has been shown to have a negative impact on your health because what someone eats will have a direct effect on their body. While playing video games, people can lose track of time, and in turn, lose track of the meals that they are supposed to be eating. For example, one man realized he had a problem with video games when “he found himself barely eating [and] living off iced tea because he didn't want to sacrifice gaming time to the time it took to prepare a meal,” (News24). Unfortunately, eating healthier food requires more time to prepare and it becomes less important to someone who would rather spend that time gaming and it becomes a problem when they start starving themselves just to gain more play time. Another danger of video game addiction is sleep deprivation. Sleep disorders and chronic sleep loss have been shown to put you at higher risk for many heart conditions. Sleep deprivation combined with an unhealthy lifestyle can actually put many gamers at an even higher risk for heart attack or stroke. One of the biggest problems with video game addiction is determining the separation between video game fiction and reality. Video game addicts “who end up in hospital with [this addiction] usually only admit themselves after they've lost nearly everything: Partners and jobs, their real life,” (News24). This separation between the real world and the virtual word is the reason that we are able to maintain relationships with those around us. When someone loses that separation, they cannot tell the difference between what is real life and what is a game and it becomes not only dangerous for them, but also dangerous for others.

\paragraph{}Gamers who exhibit aggressive behavior are among the most dangerous types of addicts. Aggressive outbursts from an addicted gamer have two main sources: time away from the game and determining the difference between real life the game. Although someone differentiating between real life and the game is more rare, there have been thousands of documented cases where this form of aggression has been called into question, such in the case of Daniel Petric. During 2008, in Ohio, 16 year old Petric became heavily addiction to his Xbox after being housebound for over a year while he was recovering from a staph infection. Petric went against his parents’ will and purchased Halo 3, a popular violent video game. Halo 3 was the most anticipated video game of 2007, however it is a far cry from family friendly, “this game lets gamers shoot aliens from a first-person perspective with the gun seen on the screen at all times... blood that splatters can be alien or human,” (Saltzman). Petric would then spend up to 13 hours a day playing while his parents were at work. One afternoon, Petric’s father caught his son playing the game and took it away to place in his gun safe. Due to his anger towards his father for taking the game, Petric took the keys to his father’s safe and stole both the game and the father’s handgun. Petric then proceeded to shoot and kill his mother and severely wound his father. 

\paragraph{}In some cases, the physical effects on the body due to video game addiction can also be deadly. Chen Rong-Yu was a frequent customer at an internet cafe in Taiwan last year. He would often stay and play video games online for days at a time. Unfortunately, Yu suffered from a heart condition, and although he was receiving treatment, he chose to ignore his pain while he marathoned 23 hours of League of Legends. When he went to stand up, Yu suffered a major heart attack caused by a blood clot that had developed in his legs and he passed away. League of Legends is an online multiplayer game that has become popular across the globe. Researchers of video game addiction in both Korea and the United States have “reported specifically that Massive Multiplayer Online Role Playing Games (MMORPGs) are the main culprits in cases of online video game addiction,” (van Rooij 205), because the game can never be completed. Online video games are constantly being updated so players are forced to continue playing to keep up with the game’s story. 

\paragraph{}Daniel Petric’s brain was still in the process of developing when he became addicted, and his addiction left him with only a need to continue playing the game, no matter what the cost. Similar patterns of addiction that are seen in adolescents can also develop in adults. Yu was just 23 when he died while playing League of Legends. His addiction to the game manifested itself after his adolescence, however it still caused him to avoid all social obligations and, in the end, killed him.	

\paragraph{}Video game addiction is a problem for men and women of any age worldwide and, unlike gambling or alcoholism, society does not see it as problematic. However, research has shown some of the dangers that video games can cause. People have already died from their addictions, thus making it no different than an addiction to drugs or alcohol, and video game addiction needs to start being taken seriously. The longer that this goes unrecognized, the more of a problem it will become.

\section{}
Dr.Conrad, Brent (24 Nov. 2013) "How to Help Children Addicted to Video Games - A Guide for Parents. Retrieved from http://www.techaddiction.ca/children-addicted-to-video-games.html

Mikelberg, Amanda (24 Nov. 2013) "Corpse of League of Legends Player Ignored at Internet Cafe for Nine Hours." NY Daily News. Retrieved from http://www.nydailynews.com/news/world/corpse-league-legends-player-internet-cafe-hours-article-1.1017013

(24 Nov. 2013) "News24 News. Breaking News. First." Retrieved from http://www.news24.com/Technology/News/Dangers-of-gaming-addiction-20130113

Saltzman, Mark. "Halo 3." (24 Nov 2013) Turner, Karl. "17-year-old Accused of Killing Mother over Halo 3 Video Game May Get Verdict Soon." Retrieved from http://blog.cleveland.com/metro/2008/12/trial\_of\_boy\_accused\_of\_killin.html
=======
\paragraph{}
There are also several dangers of becoming addicted to video games and when left untreated can result in physical harm and, in extreme cases, death. One of the dangers of video game addiction is eating disorders. For decades, unhealthy eating has been shown to have a negative impact on your health because what someone eats will have a direct effect on their body. While playing video games, people can lose track of time, and in turn, lose track of the meals that they are supposed to be eating. For example, one man realized he had a problem with video games when “he found himself barely eating [and] living off iced tea because he didn't want to sacrifice gaming time to the time it took to prepare a meal,” (News24). Unfortunately, eating healthier food requires more time to prepare and it becomes less important to someone who would rather spend that time gaming and it becomes a problem when they start starving themselves just to gain more play time. Another danger of video game addiction is sleep deprivation. Sleep disorders and chronic sleep loss have been shown to put you at higher risk for many heart conditions. Sleep deprivation combined with an unhealthy lifestyle can actually put many gamers at an even higher risk for heart attack or stroke. One of the biggest problems with video game addiction is determining the separation between video game fiction and reality. Video game addicts “who end up in hospital with [this addiction] usually only admit themselves after they've lost nearly everything: Partners and jobs, their real life,” (News24). This separation between the real world and the virtual word is the reason that we are able to maintain relationships with those around us. When someone loses that separation, they cannot tell the difference between what is real life and what is a game and it becomes not only dangerous for them, but also dangerous for others.
\paragraph{}
Gamers who exhibit aggressive behavior are among the most dangerous types of addicts. Aggressive outbursts from an addicted gamer have two main sources: time away from the game and determining the difference between real life the game. Although someone differentiating between real life and the game is more rare, there have been thousands of documented cases where this form of aggression has been called into question, such in the case of Daniel Petric. During 2008, in Ohio, 16 year old Petric became heavily addiction to his Xbox after being housebound for over a year while he was recovering from a staph infection. Petric went against his parents’ will and purchased Halo 3, a popular violent video game. Halo 3 was the most anticipated video game of 2007, however it is a far cry from family friendly, “this game lets gamers shoot aliens from a first-person perspective with the gun seen on the screen at all times... blood that splatters can be alien or human,” (Saltzman). Petric would then spend up to 13 hours a day playing while his parents were at work. One afternoon, Petric’s father caught his son playing the game and took it away to place in his gun safe. Due to his anger towards his father for taking the game, Petric took the keys to his father’s safe and stole both the game and the father’s handgun. Petric then proceeded to shoot and kill his mother and severely wound his father. 
\paragraph{}
In some cases, the physical effects on the body due to video game addiction can also be deadly. Chen Rong-Yu was a frequent customer at an internet cafe in Taiwan last year. He would often stay and play video games online for days at a time. Unfortunately, Yu suffered from a heart condition, and although he was receiving treatment, he chose to ignore his pain while he marathoned 23 hours of League of Legends. When he went to stand up, Yu suffered a major heart attack caused by a blood clot that had developed in his legs and he passed away. League of Legends is an online multiplayer game that has become popular across the globe. Researchers of video game addiction in both Korea and the United States have “reported specifically that Massive Multiplayer Online Role Playing Games (MMORPGs) are the main culprits in cases of online video game addiction,” (van Rooij 205), because the game can never be completed. Online video games are constantly being updated so players are forced to continue playing to keep up with the game’s story. 
\paragraph{}
Daniel Petric’s brain was still in the process of developing when he became addicted, and his addiction left him with only a need to continue playing the game, no matter what the cost. Similar patterns of addiction that are seen in adolescents can also develop in adults. Yu was just 23 when he died while playing League of Legends. His addiction to the game manifested itself after his adolescence, however it still caused him to avoid all social obligations and, in the end, killed him.	
\paragraph{}
Video game addiction is a problem for men and women of any age worldwide and, unlike gambling or alcoholism, society does not see it as problematic. However, research has shown some of the dangers that video games can cause. People have already died from their addictions, thus making it no different than an addiction to drugs or alcohol, and video game addiction needs to start being taken seriously. The longer that this goes unrecognized, the more of a problem it will become.

\section{Make these citations correct.....}
\paragraph{}
Conrad, Brent, Dr. "How to Help Children Addicted to Video Games - A Guide for Parents - TechAddiction." How to Help Children Addicted to Video Games - A Guide for Parents - TechAddiction. N.p., n.d. Web. 24 Nov. 2013. <http://www.techaddiction.ca/children-addicted-to-video-games.html>.

Mikelberg, Amanda. "Corpse of League of Legends Player Ignored at Internet Cafe for Nine Hours." NY Daily News. N.p., 4 Feb. 2012. Web. 24 Nov. 2013. <http://www.nydailynews.com/news/world/corpse-league-legends-player-internet-cafe-hours-article-1.1017013>.
"News24 News. Breaking News. First." News24. N.p., n.d. Web. 24 Nov. 2013. <http://www.news24.com/Technology/News/Dangers-of-gaming-addiction-20130113>.

Saltzman, Mark. "Halo 3." Commonsensemedia.org. N.p., n.d. Web. 24 Nov. 2013. <http://www.commonsensemedia.org/game-reviews/halo-3>.
Turner, Karl. "17-year-old Accused of Killing Mother over Halo 3 Video Game May Get Verdict Soon." Cleveland.com. Sun News, 16 Dec. 2008. Web. 24 Nov. 2013. %<http://blog.cleveland.com/metro/2008/12/trial_of_boy_accused_of_killin.html>.

Van Rooij, Antonius J., Tim M. Schoenmakers, Ad A. Vermulst, Regina J.J.M. Van Den Eijnden, and Dike Van De Mheen. "Online Video Game Addiction: Identification of Addicted Adolescent Gamers." Addiction. Vol. 106. N.p.: n.p., 2010. 205-12. Print.
>>>>>>> a4ea45151d9cc614ed59d22f1c219986701ec1c5

Van Rooij, Antonius J., Tim M. Schoenmakers, Ad A. Vermulst, Regina J.J.M. Van Den Eijnden, and Dike Van De Mheen. (2010) "Online Video Game Addiction: Identification of Addicted Adolescent Gamers." Addiction. Vol. 106.

\end{document}

%
% Please see the package documentation for more information
% on the APA6 document class:
%
% http://www.ctan.org/pkg/apa6
%