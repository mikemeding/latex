%%%%%%%%%%%%%%%%%%%%%%%%%%%%%%%%%%%%%%%%%
% Arsclassica Article
% LaTeX Template
% Version 1.1 (10/6/14)
% % This template has been downloaded from:
% http://www.LaTeXTemplates.com
%
% Original author:
% Lorenzo Pantieri (http://www.lorenzopantieri.net) with extensive modifications by:
% Vel (vel@latextemplates.com)
%
% License:
% CC BY-NC-SA 3.0 (http://creativecommons.org/licenses/by-nc-sa/3.0/)
%
%%%%%%%%%%%%%%%%%%%%%%%%%%%%%%%%%%%%%%%%%

%----------------------------------------------------------------------------------------
%	PACKAGES AND OTHER DOCUMENT CONFIGURATIONS
%----------------------------------------------------------------------------------------

\documentclass[
10pt, % Main document font size
a4paper, % Paper type, use 'letterpaper' for US Letter paper
oneside, % One page layout (no page indentation)
%twoside, % Two page layout (page indentation for binding and different headers)
headinclude,footinclude, % Extra spacing for the header and footer
BCOR5mm, % Binding correction
]{scrartcl}

%%%%%%%%%%%%%%%%%%%%%%%%%%%%%%%%%%%%%%%%%
% Arsclassica Article
% Structure Specification File
%
% This file has been downloaded from:
% http://www.LaTeXTemplates.com
%
% Original author:
% Lorenzo Pantieri (http://www.lorenzopantieri.net) with extensive modifications by:
% Vel (vel@latextemplates.com)
%
% License:
% CC BY-NC-SA 3.0 (http://creativecommons.org/licenses/by-nc-sa/3.0/)
%
%%%%%%%%%%%%%%%%%%%%%%%%%%%%%%%%%%%%%%%%%

%----------------------------------------------------------------------------------------
%	REQUIRED PACKAGES
%----------------------------------------------------------------------------------------


\usepackage[T1]{fontenc}
\usepackage[utf8]{inputenc}
\usepackage{graphicx}
\usepackage{xcolor}

\renewcommand\familydefault{\sfdefault}
\usepackage{tgheros}
\usepackage[defaultmono]{droidmono}

\usepackage{amsmath,amssymb,amsthm,textcomp}
\usepackage{enumerate}
\usepackage{multicol}
\usepackage{tikz}

\usepackage{geometry}
\geometry{total={210mm,297mm},
left=25mm,right=25mm,%
bindingoffset=0mm, top=20mm,bottom=20mm}


\linespread{1.3}

\newcommand{\linia}{\rule{\linewidth}{0.5pt}}

% custom theorems if needed
\newtheoremstyle{mytheor}
    {1ex}{1ex}{\normalfont}{0pt}{\scshape}{.}{1ex}
    {{\thmname{#1 }}{\thmnumber{#2}}{\thmnote{ (#3)}}}

\theoremstyle{mytheor}
\newtheorem{defi}{Definition}


%----------------------------------------------------------------------------------------
%	THEOREM STYLES
%---------------------------------------------------------------------------------------

\theoremstyle{definition} % Define theorem styles here based on the definition style (used for definitions and examples)
\newtheorem{definition}{Definition}

\theoremstyle{plain} % Define theorem styles here based on the plain style (used for theorems, lemmas, propositions)
\newtheorem{theorem}{Theorem}

\theoremstyle{remark} % Define theorem styles here based on the remark style (used for remarks and notes)
 % Include the structure.tex file which specified the document structure and layout

\hyphenation{Fortran hy-phen-ation} % Specify custom hyphenation points in words with dashes where you would like hyphenation to occur, or alternatively, don't put any dashes in a word to stop hyphenation altogether

%----------------------------------------------------------------------------------------
%	TITLE AND AUTHOR(S)
%----------------------------------------------------------------------------------------

\title{\normalfont\spacedallcaps{Impact of the Raspberry Pi on the computing world}} % The article title

\author{\spacedlowsmallcaps{Michael Meding }} % The article author(s) - author affiliations need to be specified in the AUTHOR AFFILIATIONS block

\date{} % An optional date to appear under the author(s)

%----------------------------------------------------------------------------------------

\begin{document}

%----------------------------------------------------------------------------------------
%	HEADERS
%----------------------------------------------------------------------------------------

\renewcommand{\sectionmark}[1]{\markright{\spacedlowsmallcaps{#1}}} % The header for all pages (oneside) or for even pages (twoside)
%\renewcommand{\subsectionmark}[1]{\markright{\thesubsection~#1}} % Uncomment when using the twoside option - this modifies the header on odd pages
\lehead{\mbox{\llap{\small\thepage\kern1em\color{halfgray} \vline}\color{halfgray}\hspace{0.5em}\rightmark\hfil}} % The header style

\pagestyle{scrheadings} % Enable the headers specified in this block

%----------------------------------------------------------------------------------------
%	TABLE OF CONTENTS & LISTS OF FIGURES AND TABLES
%----------------------------------------------------------------------------------------

\maketitle % Print the title/author/date block

\setcounter{tocdepth}{2} % Set the depth of the table of contents to show sections and subsections only

\tableofcontents % Print the table of contents

%\listoffigures % Print the list of figures

%\listoftables % Print the list of tables

%----------------------------------------------------------------------------------------
%	ABSTRACT
%----------------------------------------------------------------------------------------

\section*{Abstract} % This section will not appear in the table of contents due to the star (\section*)

\paragraph{}
From humble beginnings the Raspberry Pi has been one of the fastest growing and now most popular computing platform for hobbyists, students, and businesses alike. Its \$35 dollar price tag and easy availability online has brought the world of computer science out of a cloud of mystery and into a domain where even elementary school kids can enjoy it.
This paper seeks to explore not only what the Raspberry Pi is but how it has impacted the Computer Science community as a whole.

%----------------------------------------------------------------------------------------
%	AUTHOR AFFILIATIONS
%----------------------------------------------------------------------------------------

{\let\thefootnote\relax\footnotetext{* \textit{Department of Computer Science, University of Massachusetts Lowell, United States}}}

%----------------------------------------------------------------------------------------

\newpage % Start the article content on the second page, remove this if you have a longer abstract that goes onto the second page

 
%------------------------------------------------
% What is the Raspberry Pi in detail
%------------------------------------------------
\section{What is the Raspberry Pi}

\paragraph{}
The Raspberry Pi is a small form factor \$35 computer designed for Computer Science education but has become so much more.
It is about the size of a credit card and is able to interface with all modern computer peripherals such as keyboards, mice, speakers, TV's, and computer monitors.
In addition to the standard set of peripherals the Raspberry Pi is also able to interface with its world in a more creative way by use of its GPIO pins.
These pins allow access to low level I/O which effectively turns the Raspberry Pi into a physically interactive computer.
\paragraph{}
Most computer users will only need a computer for a small subset of basic functions. Those functions include but are not limited to, word processing, web browsing, media viewing, and perhaps playing games.
However, the Raspberry Pi can do even more than just this, it can also interact with its environment.
When a computer is allowed to interact with its physical environment it becomes something more than a computer.
This additional interaction allows the user to create their own experience and to be creative in a whole new way.
\paragraph{}
The Raspberry Pi was created as a part of an educational charity based in the UK.
It's goals were to create a simple and easy platform for both children and adults to explore computers and Computer Science.

%------------------------------------------------
% Impact on the Computer Science world
%------------------------------------------------
\section{Impact on the Computer Science world}

\paragraph{}
For vast majority of people the biggest barrier to Computer Science is the cost of a computer.
A decent laptop or desktop computer can easily cost a thousand dollars or more which is far too expensive for most people.
For this reason the Raspberry Pi fills a much needed gap in the cheap computer market allowing for much broader access.
\paragraph{}
% impact on education
By far the largest impact the Raspberry Pi has had has been to do with education which makes sense as it was originally conceived to be an educational device.
"The number of hobbyist programmers
was dropping – and the number who were
taking computer science to A Level was
dropping,” Raspberry Pi Foundation
co-founder Robert Mullins says, speaking
about the origins of the platform. “One
problem was the teaching of ICT at school.
Children were being exposed to learning
about applications on a computer, rather
than computer science as a discipline." \cite{chrisedwards2016}
However, I doubt that the creator Eben Upton could have ever envisioned the scale of the impact that he has made while creating this little device.
It has enabled millions of students to discover the world of Computer Science and usher in a new age of computing.
Enabling access for everyone to afford a computer will allow us to be even more connected and allow us all to share our ideas world wide.

%Another impact of the Raspberry Pi is that it has helped to close the gender gap that exists in Computer Science which also positively impacts the field as a whole.
\paragraph{}
A more minor impact of the Raspberry Pi comes with the rise of the Internet of Things or IoT for short.
IoT is an idea that if a physical item could benefit from being connected to the computer then it should be.
This is seen most clearly in home automation systems which connect things like water heaters or dog bowls to the internet.
It might not seem useful at first but imagine going to the closest computer and being able to control most of the items in your house.
Now imagine the case for the dog bowl and that you often forget to feed the dog on time.
If you had it connected to the web it could automatically feed the dog at a certain time and alert you to when the food reserves are low.
This same concept applies to many devices around your home and has been impacted heavily by the raspberry pi as its cost of entry makes it ideal for those seldom used devices around your home.

\paragraph{}
% impact on hobbyists and adults
The impact of the Raspberry Pi also extends outside of education into both the business world and the more adult oriented hobby space.
A clear example of this is AstroPrint.
AstroPrint is a cloud 3D printing software that is written specifically to run on the Raspberry Pi.
This is intriguing as their entire business model revolves around the sale of a Raspberry Pi which is pre-installed with their software.
And yet this works. It works so well in fact that it has allowed AstroPrint to have a full time staff of employees.

%------------------------------------------------
% Future of the Raspberry Pi
%------------------------------------------------
\section{Future of the Raspberry Pi}

\paragraph{}
% talk about new models and why they are useful.
% talk about pi zero
The future of the Raspberry Pi seems bright and since its inception it has seen several hardware and core software updates.
Most importantly the memory was upgraded from the original 256mb to 512mb and the processor chip was also upgraded from a 700mHz base clock to a whopping 1gHz.
This has allowed the Raspberry Pi to handle HD video playback and 3D graphics rendering.
Additionally, there have been several new form factors which have been released thereby shrinking and expanding on both the Raspberry Pi's functionality and size.
Most notably is the Raspberry Pi Zero which has scaled both its form factor and price down to only \$5.

\paragraph{}
% talk about raspberry pi 3
As a final note, at the time of this writing the Raspberry Pi 3 was just released which has now brought brought the little \$35 computer to a whole new level.
It now includes integrated wireless and bluetooth and an amazing 1.2gHz processor.
It seems that this little computer has and will continue to have quite an impact on the world of Computer Science for many generations to come.


%----------------------------------------------------------------------------------------
%	BIBLIOGRAPHY
%----------------------------------------------------------------------------------------
\newpage % It looks better when references are on another page.

\renewcommand{\refname}{\spacedlowsmallcaps{References}} % For modifying the bibliography heading

\bibliographystyle{unsrt}

\bibliography{sample.bib} % The file containing the bibliography

%----------------------------------------------------------------------------------------

\end{document}
